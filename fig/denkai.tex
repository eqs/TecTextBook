\documentclass[dvipdfmx]{standalone}
\usepackage{tikz}
\usepackage{tecfig}
\usepackage{ifthen}
\usetikzlibrary{calc}
\usetikzlibrary{positioning}
\begin{document}
	\begin{tikzpicture}
		\newcommand\atama[5]{
			% (x, y, h, a, b)
			\coordinate (a) at (#1, #2);
			\coordinate (a1) at ($(a) + (-#4, 0)$);
			\coordinate (a2) at ($(a) + (+#4, 0)$);
			\coordinate (b1) at ($(a) + (-#4, #3)$);
			\coordinate (b2) at ($(a) + (+#4, #3)$);
			\path[fill=white] (a2) arc (0:-180:#4 and #5) -- ++(0, #3) arc (180:0:#4 and #5) -- cycle;
			\draw (a1) -- (b1);
			\draw (a2) -- (b2);
			\draw (a2) arc (0:-180:#4 and #5);
			\draw ($(a2) + (0, #3)$) arc (0:360:#4 and #5);
		}
		
		\node[scale=0.25](A) at (0, 0){%
			\begin{tikzpicture}
				\draw[black, line width=2.0pt] (0, 0) ellipse (1.1cm and 0.30cm);
				\node[draw, yscale=0.75, line width=1pt] at (-0.5, 0){};
				\node[draw, yscale=0.75, line width=1pt] at (+0.5, 0){};
			\end{tikzpicture}
		};%
		
		\node(B) at (0, 0.75){%
			\begin{tikzpicture}
				\draw (-0.125, 0) -- ++(0, -0.375);
				\draw (0.125, 0) -- ++(0, -0.5);
				\atama{0}{0}{0.5}{2.5mm}{1.25mm}
			\end{tikzpicture}
		};%

		\node[anchor=west] at ([xshift=-1mm]A.east){{\tiny +}};
		%\draw (A.-30) -- ++(0.5, 1) node {足の長いほうが+極です};
		
		\draw[-latex, line width=1pt] (B.west) -- ++(-0.75, -0.25);
		
	\end{tikzpicture}
\end{document}
