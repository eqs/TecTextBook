\documentclass[dvipdfmx]{standalone}
\usepackage{tikz}
\usepackage{ifthen}
\usetikzlibrary{calc}
\usetikzlibrary{circuits.logic.US, circuits.logic.IEC}
\usetikzlibrary{positioning}
\begin{document}

	\begin{tikzpicture}
		\node (circuit) at (3.5, 0) {\textbf{ORの回路記号}};
		\node (truth) at (0, 0) {\textbf{ORの真理値表}};
		\node (eq) at (3.5, 2) {\textbf{ORの論理式}};

		\node[anchor=south] at (eq.north) {%
			$X = A + B$
		};
		
		\node[anchor=south] at (circuit.north) {%
			\begin{tikzpicture}[circuit logic US]
				\node[anchor=east] (a) at (0,  0.25) {$A$};
				\node[anchor=east] (b) at (0, -0.25) {$B$};
				\coordinate (c) at ($(a)!0.5!(b)$);
				\node[or gate, right=0.75cm of c] (or) {};
				\node[anchor=east, right=0.5cm of or] (q) {$X$};

				\draw (a) -| ([xshift=-0.25cm]or.input 1) -- (or.input 1);
				\draw (b) -| ([xshift=-0.25cm]or.input 2) -- (or.input 2);
				\draw (or.output) -- (q);
			\end{tikzpicture}
		};
	
		\node[anchor=south] at (truth.north) {
			\begin{tabular}{|cc|c|}\hline
				\multicolumn{2}{|c|}{入力}	&	出力	\\ \hline
				$A$&$B$&$X$\\ \hline
				0&0&0 \\
				0&1&1 \\
				1&0&1 \\
				1&1&1 \\ \hline
			\end{tabular}
		};

	\end{tikzpicture}
\end{document}
