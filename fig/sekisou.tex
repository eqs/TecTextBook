\documentclass[dvipdfmx]{standalone}
\usepackage{tikz}
\usepackage{tecfig}
\usepackage{ifthen}
\usetikzlibrary{calc}
\usetikzlibrary{positioning}
\begin{document}
	\begin{tikzpicture}
		\node[text width=4.5cm, anchor=west](node1) at (2, 1.5)
			{数字 (型番) が書かれている\\空色のコンデンサ};
		\node[text width=4.5cm, anchor=west](node2) at (2, 0)
			{基板面には,コンデンサの\\マークの一部が白色で\\書かれている};
		\node[draw, yscale=0.75, line width=0.75pt] at (0-0.25, 0){};
		\node[draw, yscale=0.75, line width=0.75pt] at (0+0.25, 0){};
		\draw[line width=1.5pt](0-0.0625, -0.25) -- ++(0, 0.5);
		\draw[line width=1.5pt](0+0.0625, -0.25) -- ++(0, 0.5);

		\draw[line width=0.75pt](0-0.25, 0.5) -- ++(0, 0.75) coordinate(a);
		\draw[line width=0.75pt](0+0.25, 0.5) -- ++(0, 0.75) coordinate(b);

		\draw (a) -- ++(-0.1, 0.1) -- ++(0, 0.45) 
			-- ++(0.1, 0.1) -- ++(0.5, 0) -- ++(0.1, -0.1) -- ++(0, -0.45) -- ++(-0.1, -0.1)
			-- ++(-0.1, 0.1) -- ++(-0.3, 0) -- ++(-0.1, -0.1);
		\draw[-latex] (node2.west) to[bend left] ++(-1.75, -0.25);
	\end{tikzpicture}
\end{document}
