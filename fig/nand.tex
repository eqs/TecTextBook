\documentclass[dvipdfmx]{standalone}
\usepackage{tecfig}
\begin{document}

	\begin{tikzpicture}
		\node (circuit) at (3.75, 0) {\textbf{NANDの回路記号}};
		\node (truth) at (0, 0) {\textbf{NANDの真理値表}};
		\node (eq) at (3.75, 3.75) {\textbf{NANDの論理式}};

		\node[anchor=south] at (eq.north) {%
			$X = \overline{A \cdot B}$
		};
		
		\node[anchor=south](nor node) at (circuit.north) {%
			\begin{tikzpicture}[circuit logic US]
				\node[anchor=east] (a) at (0,  0.25) {$A$};
				\node[anchor=east] (b) at (0, -0.25) {$B$};
				\coordinate (c) at ($(a)!0.5!(b)$);
				\node[nand gate, right=0.75cm of c] (nand) {};
				\node[anchor=east, right=0.5cm of nand] (q) {$X$};

				\draw (a) -| ([xshift=-0.25cm]nand.input 1) -- (nand.input 1);
				\draw (b) -| ([xshift=-0.25cm]nand.input 2) -- (nand.input 2);
				\draw (nand.output) -- (q);
			\end{tikzpicture}
		};
		
		\node[above=0.5cm of nor node] (not or) {%
			\begin{tikzpicture}[circuit logic US]
				\node[anchor=east] (a) at (0,  0.25) {$A$};
				\node[anchor=east] (b) at (0, -0.25) {$B$};
				\coordinate (c) at ($(a)!0.5!(b)$);
				\node[and gate,  right=0.75cm of c] (and) {};
				\node[not gate, right=0.5cm of and] (not) {};
				\node[anchor=east, right=0.5cm of not] (q) {$X$};

				\draw (a) -| ([xshift=-0.25cm]and.input 1) -- (and.input 1);
				\draw (b) -| ([xshift=-0.25cm]and.input 2) -- (and.input 2);
				\draw (and.output) -- (not.input);
				\draw (not.output) -- (q);
			\end{tikzpicture}
		};

		\draw (not or) circle[x radius=2.25cm, y radius=0.75cm];
		\draw[-latex, line width=1pt] (not or) ++(0, -0.80) -- ++(0, -0.5);
	
		\node[anchor=south] at (truth.north) {
			\begin{tabular}{|cc|c|}\hline
				\multicolumn{2}{|c|}{入力}	&	出力	\\ \hline
				$A$&$B$&$X$\\ \hline
				0&0&1 \\
				0&1&1 \\
				1&0&1 \\
				1&1&0 \\ \hline
			\end{tabular}
		};

	\end{tikzpicture}
\end{document}
