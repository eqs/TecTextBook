\documentclass[dvipdfmx]{standalone}
\usepackage{tikz}
\usepackage{tecfig}
\usepackage{ifthen}
\usetikzlibrary{calc}
\usetikzlibrary{positioning}
\begin{document}
	\begin{tikzpicture}
		\newcommand\atama[5]{
			% (x, y, h, a, b)
			\coordinate (a) at (#1, #2);
			\coordinate (a1) at ($(a) + (-#4, 0)$);
			\coordinate (a2) at ($(a) + (+#4, 0)$);
			\coordinate (b1) at ($(a) + (-#4, #3)$);
			\coordinate (b2) at ($(a) + (+#4, #3)$);
			\path[fill=white] (a2) arc (0:-180:#4 and #5) -- ++(0, #3) arc (180:0:#4 and #5) -- cycle;
			\draw[black] (a1) -- (b1);
			\draw[black] (a2) -- (b2);
			\draw[black] (a2) arc (0:-180:#4 and #5);
			\draw[black] ($(a) + (0, #3)$) ellipse (#4 and #5);
			\draw[black] ($(a) + (0, #3)$) ellipse (#4/6 and #5/6);
		}
		
		\node[text width=3cm] at (0, 0.5)
			{円筒形の部品です。\\ 向きはありません。};
		\draw (3 - 0.5, 0) -- ++(0, -0.5);
		\draw (3 + 0.5, 0) -- ++(0, -0.55);
		\atama{3}{0}{0.75}{1cm}{0.25cm}
		\draw[black, line width=1.5pt] (3, -1) ellipse (1.1cm and 0.30cm);
		\node[draw, yscale=0.5, line width=1pt] at (3-0.5, -1){};
		\node[draw, yscale=0.5, line width=1pt] at (3+0.5, -1){};
		
	\end{tikzpicture}
\end{document}
