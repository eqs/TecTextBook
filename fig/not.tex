\documentclass[dvipdfmx]{standalone}
\usepackage{tikz}
\usepackage{ifthen}
\usetikzlibrary{calc}
\usetikzlibrary{circuits.logic.US, circuits.logic.IEC}
\usetikzlibrary{positioning}
\begin{document}

	\begin{tikzpicture}
		\node (circuit) at (3.5, 0) {\textbf{NOTの回路記号}};
		\node (truth) at (0, 0) {\textbf{NOTの真理値表}};
		\node (eq) at (3.5, 2.5) {\textbf{NOTの論理式}};

		\node[anchor=south] at (eq.north) {%
			$X = \overline A$
		};
		
		\node[anchor=south] at (circuit.north) {%
			\begin{tikzpicture}[circuit logic US]
				\node[anchor=east] (a1) at (0,  1.0) {$A$};
				\node[anchor=east] (a2) at (0,  0.0) {$A$};
				\node[not gate, right=0.5cm of a1] (not1) {};
				\node[buffer gate, right=0.5cm of a2, logic gate inputs=i] (not2) {};
				\node[anchor=west, right=0.5cm of not1] (q1) {$X$};
				\node[anchor=west, right=0.5cm of not2] (q2) {$X$};

				\draw (a1) -- (not1.input) (not1.output) -- (q1);
				\draw (a2) -- (not2.input) (not2.output) -- (q2);
			\end{tikzpicture}
		};
	
		\node[anchor=south] at (truth.north) {
			\begin{tabular}{|c|c|}\hline
				入力	&	出力	\\ \hline
				$A$&$X$\\ \hline
				0&1 \\
				1&0 \\ \hline
			\end{tabular}
		};

	\end{tikzpicture}
\end{document}
