\documentclass[dvipdfmx]{standalone}
\usepackage{tikz}
\usepackage{tecfig}
\usepackage{ifthen}
\usetikzlibrary{calc}
\usetikzlibrary{positioning}
\begin{document}
	\begin{tikzpicture}
		\newcommand\atama[5]{
			% (x, y, h, a, b)
			\coordinate (a) at (#1, #2);
			\coordinate (a1) at ($(a) + (-#4, 0)$);
			\coordinate (a2) at ($(a) + (+#4, 0)$);
			\coordinate (b1) at ($(a) + (-#4, #3)$);
			\coordinate (b2) at ($(a) + (+#4, #3)$);
			\path[fill=white] (a2) arc (0:-180:#4 and #5) -- ++(0, #3) arc (180:0:#4 and #5) -- cycle;
			\draw (a1) -- (b1);
			\draw (a2) -- (b2);
			\draw (a2) arc (0:-180:#4 and #5);
			\draw ($(a2) + (0, #3)$) arc (0:360:#4 and #5);
		}
		\node[anchor=west] at (1.0, 3){スイッチの頭};
		\node[anchor=west] at (1.0, 2){しっかり押し込む};
		\node[anchor=west] at (1.0, 0){プッシュスイッチ本体};
		
		% スイッチ本体を書く
		\draw (-0.75, -0.75) rectangle ++(1.5, 1.5);
		\draw (-0.75,  0.75) -- ++(0.25, 0.5) coordinate (c1);
		\draw (+0.75,  0.75) -- ++(0.25, 0.5) coordinate (c2);
		\draw (+0.75, -0.75) -- ++(0.25, 0.5) coordinate (c3);
		\coordinate (c4) at (-0.75, -0.75);
		\coordinate (c5) at (+0.75, -0.75);
		\draw (c1) -- (c2) -- (c3);

		\draw (c3) -- ++(0, -0.2);
		\draw (c4) -- ++(0, -0.2);
		\draw (c5) -- ++(0, -0.2);

		\draw[-latex, dotted] (0.125, 3) -- ++(0, -0.9);
		
		\atama{0.125}{2.875}{0.25}{4.5mm}{2.5mm}
		\atama{0.125}{1.125}{0.25}{5mm}{2.5mm}
		\atama{0.125}{1.375}{0.5}{2.5mm}{1.25mm}
		
	\end{tikzpicture}
\end{document}
