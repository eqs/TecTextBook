\renewcommand{\myepsfbox}[1]{\epsfbox{chap2/#1}}

\chapter{情報の表現}
この章では,コンピュータ内部でどのように情報が表現されているのか,
その情報をどのような回路で扱うことができるのか,簡単に紹介します.

\section{コンピュータ内部の情報表現}
人は,情報を音声や文字,絵等で表現することができます.
コンピュータも表面的には音声や文字,絵を扱うことができますが,
コンピュータの内部は電子回路で構成されているので,最終的には
電圧や電流で情報を表現せざるを得ません.

\begin{center}
\fbox{\parbox{7cm}{
\[ 情報の表現  =  \left(
 \begin{array}{l}
  人 : 音声,文字,絵,... \\
  コンピュータ : 電圧,電流 \\
 \end{array} \right)
\]
}}
\end{center}

\subsection{電気を用いた情報の表現}
電圧や電流で情報表現する方法には,いろいろなアイデアがあります.
しかし,現在のコンピュータは,電圧が{\bf ある/ない(ON/OFF)}か,
電流が{\bf 流れている/流れていない(ON/OFF)}のような,
二つの状態だけを用いて情報を表現する方法を使っています.
(デジタル・コンピュータ)

\begin{center}
\fbox{\parbox{7cm}{
コンピュータの情報表現 \\
 電圧が{\bf ある/ない}\\
 電流が{\bf 流れる/流れない}\\
   ↓\\
 ONとOFFの二つの状態だけを用いて\\
 情報を表現する.(回路が作りやすい)
}}
\end{center}

次の例を見て下さい.電気の「ON/OFF」で「おおかみが来たか/来ていないか」の
どちらの状態なのかを伝達できる「情報の表示装置」を実現することができます.
電気の「ON/OFF」を用いて情報を表現することができました.

\begin{center}
\fbox{\parbox{7cm}{
{\bf 例 おおかみが来た情報表示装置} \\
 ランプ点灯:おおかみが来た\\
 ランプ消灯:おおかみが来ていない\\
\vspace{0.2cm}
\epsfxsize=7cm
%\epsfysize=22.5cm
\myepsfbox{ookami.pdf}
}}
\end{center}

\subsection{ビット}

前の例では,ランプの「ON/OFF」を用いて「二つの状態のどちらなのか」を表し
ました.このような,「二つのどちらか」を表す情報が「情報の最小単位」
になります.情報の最小単位のことを{\bf 「ビット(bit)」}と呼びます.

\begin{center}
\fbox{\parbox{7cm}{
{\bf on/offのどちらか → 情報の最小単位(ビット)}
}}
\end{center}

通常,ビットの値は「ON/OFF」ではなく,「1/0」で書きます.
\begin{center}
{\small
\fbox{\parbox{7cm}{
\[\left(
 \begin{array}{l}
 ON : 1 \\
OFF : 0
 \end{array} \right)
\]
}}}
\end{center}

「おおかみが来た情報」は,ビットの値に次のように対応付けできます.

\begin{center}
\begin{tabular}{|l|l|} \hline
ビット値 & おおかみが来た情報 \\
\hline
0(off) & おおかみがきていない \\
1(on)  & おおかみが来た!!! \\
\hline
\end{tabular}
\end{center}

\subsection{より複雑な情報の表現}

二つの状態では表現できない,
もっと複雑な情報はどうやって表現したら良いでしょうか.

前の例で,やって来たおおかみの頭数により牧場の人が対応を変化させた
い場合が考えられます.そのためには,「来たか/来なかったか」だけの
情報だけでは十分ではなく,{\bf 「たくさん来たか」}を知らせる必要が
あります.それには,複数のビットを組み合わせて使用します.

以下に,複数のビットの組み合わせて表現した「拡張おおかみが来た情報」
を示します.

\begin{center}
\begin{tabular}{|c|l|l} \cline{1-2}
ビット値 & 拡張おおかみが来た情報 & \\
\cline{1-2}
00 & おおかみがきていない   & 平気 \\
01 & おおかみが1頭来た     & 戦う\\
10 & おおかみが2頭来た     & ?\\
11 & おおかみがたくさん来た & 逃げる \\
\cline{1-2}
\end{tabular}
\end{center}

\begin{center}
\fbox{\parbox{7cm}{
{\bf 例 拡張おおかみが来た情報表示装置} \\
\vspace{0.2cm}
\epsfxsize=7cm
%\epsfysize=22.5cm
\myepsfbox{ookami2.pdf}
}}
\end{center}

この例で見たように,2ビットを用いれば4種類の情報を表現することが
できます.

一般に,nビットを用いると$2^n$種類の情報を表現することができます.
システム内で必要なビット数を決めて,それの組合せに意味付けをすれば,
どんな情報だって表現できます.

{\footnotesize
\begin{center}
\begin{tabular}{|c|l|c|} \hline
ビット数 & \multicolumn{1}{c|}{ビットの組合せ} & 組合せ数\\
\hline
1 & 0 1   & 2 \\
2 & 00 01 10 11 & 4 \\
3 & 000 001 010 011 100 101 110 111 & 8 \\
...& ... &\\
n &  & $2^n$ \\
\hline
\end{tabular}
\end{center}
}

ビットだけでは情報の単位として小さすぎるので,4ビットまとめたもの,
8ビットまとめたものにも名前があります.

{\small
\begin{center}
\fbox{\parbox{7cm}{
\begin{tabular}{l l l}
「4ビット」& = &「1ニブル」 \\
「8ビット」& = & 「1バイト」 \\
\end{tabular}
}}
\end{center}
}

\section{数値の表現}

ビットの組合せをどのように意味付けするかは,前節の例のように
「システムが扱う必要のある情報」により,毎回,約束すればよいのです.
しかし,どのコンピュータでも同じ方法で意味付けされている情報もあり
ます.それの一つが数値の表現方法です.

\subsection{2進数}

コンピュータの内部では,数値を2進数で表すのが普通です.
我々が普段使用している10進数と,2進数の特徴比較を次に示します.

{\small
\begin{center}
\fbox{\parbox{7cm}{
{\bf 10進数と2進数の比較} \\
 10進数の特徴 \\
  (1) 0 〜 9の10種類の数字を使用する.\\
  (2) 1桁毎に10倍の重みをもつ.\\
 2進数の特徴 \\
  (1) 0と1の2種類の数字を使用する.\\
  (2) 1桁毎に2倍の重みをもつ.\\
}}
\end{center}
}

コンピュータの内部では,ビットが情報の表現に使用されています.
そこで,ビット値の0/1をそのまま2進数の1桁と考えれば数値が表現できます.

例えば,4ビット用いると0 〜 15の数が次のように表現できます.

\begin{center}
{\footnotesize
\begin{tabular}{|c|c|c|c|l }
\cline{1-4}
{\footnotesize ビット3}&
{\footnotesize ビット2}&
{\footnotesize ビット1}&
{\footnotesize ビット0}&意味\\
($b_3$)&($b_2$)&($b_1$)&($b_0$)&\\
\cline{1-4}
 0 &  0 &  0 &  0 &  0 \\
 0 &  0 &  0 &  1 &  1 \\
 0 &  0 &  1 &  0 &  2 \\
 0 &  0 &  1 &  1 &  3 \\
 0 &  1 &  0 &  0 &  4 \\
 0 &  1 &  0 &  1 &  5 \\
 0 &  1 &  1 &  0 &  6 \\
 0 &  1 &  1 &  1 &  7 \\
 1 &  0 &  0 &  0 &  8 \\
 1 &  0 &  0 &  1 &  9 \\
 1 &  0 &  1 &  0 & 10 \\
 1 &  0 &  1 &  1 & 11 \\
 1 &  1 &  0 &  0 & 12 \\
 1 &  1 &  0 &  1 & 13 \\
 1 &  1 &  1 &  0 & 14 \\
 1 &  1 &  1 &  1 & 15 \\
\cline{1-4}
\end{tabular}
}
\end{center}

一般にnビットで$0~〜~2^n-1$の範囲の数を表現することができます.

\subsection{2進数と10進数の相互変換}

前の4ビットの例なら,2進数と10進数の対応を暗記することが可能です.
しかし,8ビットの場合ならどうでしょう?

組合せは256もあり,とても暗記できそうにありません.対応を計算で
求める必要があります.

\begin{enumerate}
\item 2進から10進への変換

2進数の桁ごとの重みは,桁の番号をnとすると$2^n$になります.
\begin{center}
\begin{tabular}{c c c c c }
$b_3$ & $b_2$ & $b_1$ & $b_0$ \\
$2^3 = 8$ & $2^2 = 4$ & $2^1 = 2$ & $2^0 = 1$ \\
\end{tabular}
\end{center}
2進数の数値は,その桁の重みと桁の値を掛け合わせたものの合計です.
例えば2進数の$1010_2$は,$2^3$の桁が1,$2^2$の桁が0,
$2^1$の桁が1,$2^0$の桁が0ですから,
次のように計算できます.
{\small
\begin{eqnarray}
1010_2 = 1 \times 2^3 + 0 \times 2^2 + 1 \times 2^1 + 0 \times 2^0 \nonumber \\
 = 1 \times 8 + 0 \times 4 + 1 \times 2 + 0 \times 1 \nonumber \\
 = 8 + 0 + 2 + 0 \nonumber \\
 = 10_{10} \nonumber
\end{eqnarray}
}

\item 10進から2進への変換

次に,10進数を2進数に変換する方法を考えます.ここで着目するのは
桁の移動です.

10進数では,値を10で割ると右に1桁移動します.2進数では,2で
割ると右に1桁移動します.どちらの場合でも,割算をした時の余りは,最
下位の桁からはみ出した数になります.

つまり,数値を2で割った時の余りは2進数を右に1桁移動したときはみ出
してきた数を表しています.

そこで,2で割る操作を繰り返しながらはみ出して来た数を記録すれば,
もとの数を2進数で表したときの0/1の並びが分かります.

\begin{center}
\begin{tabular}{l r l l }
      & $12_{10}$ & = & $1100_2$ \\
$1/2$↓ &                   &                         \\
      & $6_{10}^{\cdots 0}$  & = & $0110_2^{\cdots 0}$ \\
$1/2$↓ &                    &                         \\
      & $3_{10}^{\cdots 0}$  & = & $0011_2^{\cdots 0}$ \\
$1/2$↓ &                    &                         \\
      & $1_{10}^{\cdots 1}$  & = & $0001_2^{\cdots 1}$ \\
$1/2$↓ &                    &                         \\
      & $0_{10}^{\cdots 1}$  & = & $0000_2^{\cdots 1}$ \\
\end{tabular}
\end{center}
\end{enumerate}
\begin{center}
\fbox{\parbox{6.8cm}{
{\bf 計算方法と計算例}
\begin{quote}
$2 \underline{) ~~100 } $\\
$2 \underline{) ~~~50  } {\cdots 0}$ \\
$2 \underline{) ~~~25  } {\cdots 0}$ \\
$2 \underline{) ~~~12  } {\cdots 1}$ \\
$2 \underline{) ~~~~6  } {\cdots 0}$ \\
$2 \underline{) ~~~~3  } {\cdots 0}$ \\
$2 \underline{) ~~~~1  } {\cdots 1}$ \\
$~~  ~~~~~0 {\cdots 1}$
\end{quote}
余りを右から順に並べると $1100100_2$
}}
\end{center}

\subsection{16進数}

2進数は,桁数が多くなり書き表すのに不便です.そこで,2進数
4桁をまとめて16進数一桁で書き表します.
9より大きな数字が無いので,
16進数ではアルファベットを数字の代用にします.

\begin{center}{\small
\begin{tabular}{| c | c | c | }
\hline
{\bf 2進数} & {\bf 16進数} & {\bf 10進数} \\
\hline
$0000_2$ & $0_{16}$ & $0_{10}$ \\
$0001_2$ & $1_{16}$ & $1_{10}$ \\
$0010_2$ & $2_{16}$ & $2_{10}$ \\
$0011_2$ & $3_{16}$ & $3_{10}$ \\
$0100_2$ & $4_{16}$ & $4_{10}$ \\
$0101_2$ & $5_{16}$ & $5_{10}$ \\
$0110_2$ & $6_{16}$ & $6_{10}$ \\
$0111_2$ & $7_{16}$ & $7_{10}$ \\
$1000_2$ & $8_{16}$ & $8_{10}$ \\
$1001_2$ & $9_{16}$ & $9_{10}$ \\
$1010_2$ & $A_{16}$ & $10_{10}$ \\
$1011_2$ & $B_{16}$ & $11_{10}$ \\
$1100_2$ & $C_{16}$ & $12_{10}$ \\
$1101_2$ & $D_{16}$ & $13_{10}$ \\
$1110_2$ & $E_{16}$ & $14_{10}$ \\
$1111_2$ & $F_{16}$ & $15_{10}$ \\
\hline
\end{tabular}}
\end{center}

{\small
\begin{center}
\fbox{\parbox{7cm}{
{\bf n進数の表記}
\begin{quote}
日常生活では,数値を書き表すときは「いつも10進数」で書きます.
しかし,コンピュータの世界では2進数,10進数,16進数を場合により
使い分けます.そのため,何進数で書いてあるのか分からなくて
困ることがあります.

そこで,上の表のように数値の右に小さな字で何進数かを書き加えます.
ときには,数字の代わりに「2進数='b'」,「16進数='H'」を加えること
もあります.

例えば,「$01100100_2$を$01100100_b$」,「$64_{16}$を64H」のよう
に書きます.
\end{quote}
}}
\end{center}
}
\vspace{0.2cm}
\begin{flushleft}
{\bf 問題}
\end{flushleft}
\begin{enumerate}
\item
上の「2進数,16進数,10進数対応表」を暗記しなさい.
\item
10進数の16,50,100,127,130を,2進数(8桁),
16進数(2桁)で書き表しなさい.
\item
2進数の 00011100,00111000,11100000を
16進数(2桁),10進数で書き表しなさい.
\item
16進数の 1F と AA を
2進数(8桁)で書き表しなさい.
また,10進数で書き表しなさい.
\end{enumerate}

\section{負数の表現}
\label{minus}
拡張おおかみが来た情報表示装置では,
「表示装置の二つのビット(二つのランプ)を,あのように読む」
ことを約束しました.
つぎに,数値の場合は,
「n個のビットを2進数として読む」ことを約束しました.

今度は,負の数が必要になりました.
そこで,ビットの新しい読み方を約束します.
この節で出てくるビットは「符号付き数値」を表しています.
以下では,「符号付き数値を表すビット」をどのように読むかを説明します.

\subsection{符号付き絶対値表現}
使用できるビットのうち一つを符号を表すために使用します.
これを「符号ビット」と呼ぶことにします.
通常,符号ビットには最上位(左端)のビットを使用します.

次の例のように4ビット使用して,
$-7$ 〜 $+7$ の範囲を表すことができます.
この表現方法は分かりやすくて都合が良いのですが,
実際に使われることはあまりありません.

\begin{center}
\begin{tabular}{ |r c l | }
\hline
\multicolumn{3}{|c|}{\bf 符号付き絶対値表現(4ビット)の例}\\
\hline
$-7$ & ~~~~ & $1111_2$ \\
$-6$ & ~~~~ & $1110_2$ \\
$-5$ & ~~~~ & $1101_2$ \\
...  & ~~~~ & ...      \\
$-1$ & ~~~~ & $1001_2$ \\
$-0$ & ~~~~ & $1000_2$ \\
$+0$ & ~~~~ & $0000_2$ \\
$+1$ & ~~~~ & $0001_2$ \\
...  & ~~~~ & ...      \\
$+5$ & ~~~~ & $0101_2$ \\
$+6$ & ~~~~ & $0110_2$ \\
$+7$ & ~~~~ & $0111_2$ \\
\hline
\end{tabular}
\end{center}

\subsection{補数表現}
n桁のb進数でにおいて$b^n$からxを引いた数yをxに対する「bの補数」と言います.
n桁のb進数でにおいて$b^n-1$からxを引いた数zを
xに対する「(b-1)の補数」と言います.

{\small
\begin{eqnarray}
y = b^n - x ~~~ (yはxに対するbの補数) \nonumber\\
z = b^n - 1 - x ~~~ (zはxに対する(b-1)の補数) \nonumber
\end{eqnarray}
}

%{\small
%\begin{tabular}{l l}
%$y = b^n - x$     & (yはxに対するbの補数)\\
%$z = b^n - 1 - x$ & (zはxに対する(b-1)の補数)\\
%\end{tabular}
%}

例えば,10進数の世界で次のような例があります.

\begin{center}
\begin{tabular}{| l c r c l |}
\hline
\multicolumn{5}{|c|}{\bf 2桁の10進数で補数の例} \\
\hline
%              &   &       &   &        \\
$b=10進数$    & ~ &       &   &              \\
$n=2桁$       & ~ & $100$ &   &              \\
$b^n = 100$   & ~ & $-25$ & ~ & 75は25に対す \\
\cline{3-3}
$x = 25$      &   &  $75$ &   & る10の補数   \\
              &   &       &   &        \\
$b=10進数$    & ~ &       &   &              \\
$n=2桁$       & ~ &  $99$ &   &              \\
$b^n-1 = 99$  & ~ & $-25$  & ~ & 74は25に対す \\
\cline{3-3}
$x = 25$      &   &  $74$ &   & る9の補数   \\
%              &   &       &   &        \\
\hline
\end{tabular}
\end{center}

2進数の世界では次のような例があります.

\begin{center}
\begin{tabular}{| l c r c l |}
\hline
\multicolumn{5}{|c|}{\bf 4桁の2進数で補数の例} \\
\hline
%                &   &           &   &            \\
$b=2進数$       & ~ &          &   & $0110_2$は  \\ 
$n=4桁$         & ~ & $10000_2$ &   & $1010_2$に  \\
$b^n = 10000_2$ & ~ & $-1010_2$ & ~ & 対する      \\
\cline{3-3}
$x = 1010_2$    &   & $0110_2$  &   & \underline{2の補数} \\ 
                &   &           &   &            \\
$b=2進数$       & ~ &           &   & $0101_2$は \\
$n=4桁$         & ~ & $1111_2$  &   & $1010_2$に \\
$b^n-1 = 1111_2$& ~ & $-1010_2$  & ~ & 対する      \\
\cline{3-3}
$x = 1010_2$    &   & $0101_2$  &   & \underline{1の補数} \\
%                &   &           &   &            \\
\hline
\end{tabular}
\end{center}

\subsection{1の補数による負数の表現}

上の{\bf「4桁の2進数で補数の例」}にあるように,
$11...1_2$からもとの数(x)を引いた数をxに対する「1の補数」と呼びます.
次の表は$0$ 〜 $7$ に対する1の補数を計算したものです.

\begin{center}
\begin{tabular}{ | l | l |}
\hline
\multicolumn{2}{|c|}{\bf 4ビット2進数の1補数} \\
\hline
$0$  & $1111_2 - 0000_2 = 1111_2$ \\
$1$  & $1111_2 - 0001_2 = 1110_2$ \\
$2$  & $1111_2 - 0010_2 = 1101_2$ \\
$3$  & $1111_2 - 0011_2 = 1100_2$ \\
$4$  & $1111_2 - 0100_2 = 1011_2$ \\
$5$  & $1111_2 - 0101_2 = 1010_2$ \\
$6$  & $1111_2 - 0110_2 = 1001_2$ \\
$7$  & $1111_2 - 0111_2 = 1000_2$ \\
\hline
\end{tabular}
\end{center}

「$0001_2$に対する1の補数を$-0001_2$を表現するために使用する.」,
「$0010_2$に対する1の補数を$-0010_2$を表現するために使用する.」
つまり,
{\bf 「1の補数を負の数を表現するために使用する.」}と約束すれば,
次の表のように$-7$ 〜 $+7$の範囲の数を表現できます.

\newcommand{\h}{$\vert$}
\begin{center}
\begin{tabular}{ | l | l c c c c c c c c |}
\hline
\multicolumn{10}{|c|}{\bf 1の補数を用いた符号付き数値} \\
\hline
$-7$  & $1000_2$ &--&--&--&--&--&--&--& +\\
$-6$  & $1001_2$ &--&--&--&--&--&--&+ &\h\\
$-5$  & $1010_2$ &--&--&--&--&--&+ &\h&\h\\
$-4$  & $1011_2$ &--&--&--&--&+ &\h&\h&\h\\
$-3$  & $1100_2$ &--&--&--&+ &\h&\h&\h&\h\\
$-2$  & $1101_2$ &--&--&+ &\h&\h&\h&\h&\h\\
$-1$  & $1110_2$ &--&+ &\h&\h&\h&\h&\h&\h\\
$-0$  & $1111_2$ &+ &\h&\h&\h&\h&\h&\h&\h\\
$+0$  & $0000_2$ &+ &\h&\h&\h&\h&\h&\h&\h\\
$+1$  & $0001_2$ &--&+ &\h&\h&\h&\h&\h&\h\\
$+2$  & $0010_2$ &--&--&+ &\h&\h&\h&\h&\h\\
$+3$  & $0011_2$ &--&--&--&+ &\h&\h&\h&\h\\
$+4$  & $0100_2$ &--&--&--&--&+ &\h&\h&\h\\
$+5$  & $0101_2$ &--&--&--&--&--&+ &\h&\h\\
$+6$  & $0110_2$ &--&--&--&--&--&--&+ &\h\\
$+7$  & $0111_2$ &--&--&--&--&--&--&--& +\\
\hline
\end{tabular}
\end{center}

次のように1の補数(z)を求める式を$x=$の形に変形しても同じ形になるので,
$x$と$z$は「互いに1の補数」です.

{\small
\begin{eqnarray}
z = 2^n - 1 - x ~~~ (zはxに対する1の補数) \nonumber\\
x = 2^n - 1 - z ~~~ (xはzに対する1の補数) \nonumber
\end{eqnarray}
}

%{\small
%\begin{tabular}{l l}
%$z = 2^n - 1 - x$ & (zはxに対する1の補数)\\
%$x = 2^n - 1 - z$ & (xはzに対する1の補数)\\
%\end{tabular}
%}

なお,表で$+7$と$-7$が結んであるのは,
「互いに1の補数」であることを表すためです.
また,「互いに1の補数」の2数を見比べると
ビットの0/1が入れ替わった関係になっていることが分かります.
1の補数は引算だけではなく{\bf 「ビット反転」}でも計算できます.


1の補数を使用した方法も,実際に使われることはあまりありません.
コンピュータの内部で実際に使用されるのは,
次に説明する2の補数による表現です.

\subsection{2の補数による負数の表現}\label{chap2:2c}

前出の{\bf「4桁の2進数で補数の例」}にあるように,
$100...0_2$からもとの数(x)を引いた数をxに対する「2の補数」と呼びます.
次の表は$0$ 〜 $8$ に対する2の補数を計算したものです.
5ビットで表現さている部分もありますが,
四角で囲んだ4ビットに注目してください.

\begin{center}
\begin{tabular}{ | l | l r |}
\hline
\multicolumn{3}{|c|}{\bf 4ビット2進数の2補数} \\
\hline
$0$  & $1$\fbox{$0000$}$_2 - $\fbox{$0000$}$_2 = $ & $1$\fbox{$0000$}$_2$ \\
$1$  & $1$\fbox{$0000$}$_2 - $\fbox{$0001$}$_2 = $ & \fbox{$1111$}$_2$ \\
$2$  & $1$\fbox{$0000$}$_2 - $\fbox{$0010$}$_2 = $ & \fbox{$1110$}$_2$ \\
$3$  & $1$\fbox{$0000$}$_2 - $\fbox{$0011$}$_2 = $ & \fbox{$1101$}$_2$ \\
$4$  & $1$\fbox{$0000$}$_2 - $\fbox{$0100$}$_2 = $ & \fbox{$1100$}$_2$ \\
$5$  & $1$\fbox{$0000$}$_2 - $\fbox{$0101$}$_2 = $ & \fbox{$1011$}$_2$ \\
$6$  & $1$\fbox{$0000$}$_2 - $\fbox{$0110$}$_2 = $ & \fbox{$1010$}$_2$ \\
$7$  & $1$\fbox{$0000$}$_2 - $\fbox{$0111$}$_2 = $ & \fbox{$1001$}$_2$ \\
$8$  & $1$\fbox{$0000$}$_2 - $\fbox{$1000$}$_2 = $ & \fbox{$1000$}$_2$ \\
\hline
\end{tabular}
\end{center}

「$0001_2$に対する2の補数を$-0001_2$を表現するために使用する.」,
「$0010_2$に対する2の補数を$-0010_2$を表現するために使用する.」つまり
{\bf 「2の補数を負の数を表すために使用する.」}と約束すれば,
次の表のように$-8$ 〜 $+7$の範囲の数を表現できます.
(ただし$1000_2$は$-8$を表現することとします.
4ビットでは$+8$を表現することはできません.)

\begin{center}
\begin{tabular}{ | r | l c c c c c c c c |}
\hline
\multicolumn{10}{|c|}{\bf 2の補数を用いた符号付き数値} \\
\hline
$-8$  & $1000_2$ &  &  &  &  &  &  &  &  \\
$-7$  & $1001_2$ &--&--&--&--&--&--&--&+ \\
$-6$  & $1010_2$ &--&--&--&--&--&--&+ &\h\\
$-5$  & $1011_2$ &--&--&--&--&--&+ &\h&\h\\
$-4$  & $1100_2$ &--&--&--&--&+ &\h&\h&\h\\
$-3$  & $1101_2$ &--&--&--&+ &\h&\h&\h&\h\\
$-2$  & $1110_2$ &--&--&+ &\h&\h&\h&\h&\h\\
$-1$  & $1111_2$ &--&+ &\h&\h&\h&\h&\h&\h\\
$ 0$  & $0000_2$ &+ &\h&\h&\h&\h&\h&\h&\h\\
$ 1$  & $0001_2$ &--&+ &\h&\h&\h&\h&\h&\h\\
$ 2$  & $0010_2$ &--&--&+ &\h&\h&\h&\h&\h\\
$ 3$  & $0011_2$ &--&--&--&+ &\h&\h&\h&\h\\
$ 4$  & $0100_2$ &--&--&--&--&+ &\h&\h&\h\\
$ 5$  & $0101_2$ &--&--&--&--&--&+ &\h&\h\\
$ 6$  & $0110_2$ &--&--&--&--&--&--&+ &\h\\
$ 7$  & $0111_2$ &--&--&--&--&--&--&--&+ \\
\hline
\end{tabular}
\end{center}


次のように2の補数(y)を求める式を$x=$の形に変形しても同じ形になるので,
$x$と$y$は「互いに2の補数」です.

{\small
\begin{eqnarray}
y = 2^n - x ~~~ (yはxに対する2の補数) \nonumber\\
x = 2^n - y ~~~ (xはyに対する2の補数) \nonumber
\end{eqnarray}
}

%「2の補数を負の数を表現するために使用する.」と約束することにより,
%4ビットで $-8$ 〜 $+7$ までの数を表現することができました.

一般に,2の補数表現を用いたnビット符号付き2進数が表現できる数値の
範囲は次の式で計算できます.
{\small
\begin{eqnarray}
-2^{n-1} ~~ 〜 ~~ 2^{n-1} -1 \nonumber
\end{eqnarray}
}

2の補数表現を用いると,
2進数の足し算・引き算が,
正負のどちらの数でも同じ手順で計算できます(詳しくは後述).
「手順が同じ=演算回路が同じ」ことになりますので,
実際にコンピュータを製作する上では非常に都合がよく,
現在のコンピュータは,
ほとんどの機種で2の補数表現を採用しています.

\subsection{2の補数を求める手順}

$-x$ をnビット2の補数表現($y$)で表します.
2の補数の定義より,$y$は次のように計算できます.
{\small
\begin{eqnarray}
y = 2^n - x                  \label{equb} \\
  = (2^n - 1 - x) + 1        \nonumber    \\
  = ( xの1の補数 ) + 1       \nonumber
\end{eqnarray}
}
「xに対する1の補数」はビット反転で簡単に求めることができます.
「xに対する2の補数」は「xをビット反転した後,1を加える」ことにより
簡単に求めることができます.

\begin{center}
\fbox{ビット反転した後,1を加える.}
\end{center}

\subsection{2の補数から元の数を求める手順}
\label{hanten}

次に,逆変換について考えます.
%xに対する2の補数yからxを求める手順です.
既に\ref{chap2:2c}で触れたように2の補数と元の数は互いに「2の補数」ですから,
2の補数を求めるのと同じ計算で,2の補数から元の数を求めることができます.

%{\small
%\begin{eqnarray}
%B = 2^n - A            \nonumber \\
%0 = 2^n - A - B        \nonumber \\
%A = 2^n - B            \label{equa}
%\end{eqnarray}
%}
%(\ref{equb})式と(\ref{equa})式では,AとBが入れ換わっただけで,
%同じ型をしています.
%このことから,AからBを求める計算と,BからAを求める計算の手順が同じことが
%分かります.

\begin{center}
\fbox{ビット反転した後,1を加える.}
\end{center}

%\pagebreak

\section{2進数の計算}

ここでは,2進数の和と差の計算方法を学びます.

\subsection{正の数の計算}
2進数の計算も10進数と同じ要領です.
桁上がりが10ではなく,
2で発生することに注意してください.

もちろん,2進数で計算しても10進数で計算しても同じ計算結果になります.

\begin{center}
\fbox{\parbox{7cm}{
{\bf 2進数と10進数の計算を比較}\\

\begin{tabular}{  l r l  l r l l}
\multicolumn{2}{c}{10進数}  & ~~~  & \multicolumn{2}{c}{2進数} & \\
                     & $07$ &      &     & $0111_2$ & (10進の7) \\
                 $+$ & $05$ &      & $+$ & $0101_2$ & (10進の5) \\
\cline{1-2} \cline{4-5}
                     & $12$ &      &     & $1100_2$ & (10進の12) \\
&&&&&&\\
                     & $12$ &      &     & $1100_2$ & (10進の12) \\
                 $-$ & $05$ &      & $-$ & $0101_2$ & (10進の5) \\
\cline{1-2} \cline{4-5}
                     & $07$ &      &     & $0111_2$ & (10進の7) \\
\end{tabular}
\\ \\
何進数で計算しても同じ結果になる.
}}
\end{center}

\subsection{負の数を含む計算}

2の補数表現を用いると,正の数だけのときと同じ要領で負の数を
含む計算ができます.ここでは和の計算を例に説明します.

\subsubsection{正の数と負の数の和}
正の数($X$)と負の数($-A$)($A$の2の補数($B$))の和の計算を考えます.\\
$X + B$の計算($X + (-A)$)
\begin{enumerate}
\item 結果が負の場合($|A| > |X|$)\\
解は $-(A - X)$ になるはず!
{\small
\begin{eqnarray}
X + B = X + (2^n - A) \nonumber \\
      = 2^n - (A - X) \label{eque}
\end{eqnarray}
}
(\ref{eque})式は,正解$-(A - X)$の2の補数表現になっています.

\fbox{\parbox{6cm}{\small
例 $3+(-5)=-2$ (4ビット2の補数表現)\\
\begin{quote}
$3$を2進数に変換すると$0011_2$ \\
$-5$を2進数に変換すると$1011_2$ \\
\\
「和を計算する」 \\
$0011_2+1011_2=1110_2= -2_{10}$ \\
\end{quote}
}}

\item 結果が正またはゼロの場合($|X| \geq |A|$)\\
解は $(X - A)$ になるはず!
{\small
\begin{eqnarray}
X + B = X + (2^n - A) \nonumber \\
      = 2^n + (X - A) \label{equf} \\
      = X - A         \nonumber
\end{eqnarray}
}
(\ref{equf})式の$2^n$は,桁あふれにより結果に残らないので,
正解($X - A$)となります.

\fbox{\parbox{6cm}{\small
例 $5+(-3)=2$ (4ビット2の補数表現)\\
\begin{quote}
$5$を2進数に変換すると$0101_2$ \\
$-3$を2進数に変換すると$1101_2$ \\
\\
「和を計算する」 \\
$0101_2 + 1101_2 = \fbox{1}0010_2 =  2_{10}$ \\
(5ビット目の1は桁あふれで消滅)
\end{quote}
}}
\end{enumerate}

\subsubsection{負の数と負の数の和}
負の数($-A_1$)(2の補数($B_1$))と\\
負の数($-A_2$)(2の補数($B_2$))の和\\
$B_1 + B_2$の計算($(-A_1) + (-A_2)$)\\
解は$-(A_1 + A_2)$になるはず!
{\small
\begin{eqnarray}
B_1 + B_2 = (2^n - A_1) + (2^n - A_2) \nonumber \\
      = 2^n + (2^n - (A_1 + A_2))     \label{equc} \\
      = 2^n - (A_1 + A_2)             \label{equd}
\end{eqnarray}
}
(\ref{equc})式の最初の$2^n$は,桁あふれにより消滅する.
(\ref{equd})式は正解($-(A_1 + A_2)$)の2の補数表現になっている.


\vspace{0.2cm}
\fbox{\parbox{6.5cm}{\small
例 $(-1)+(-3)=(-4)$ (4ビット2の補数表現)\\
\begin{quote}
$-1$を2進数に変換すると$1111_2$ \\
$-3$を2進数に変換すると$1101_2$ \\
\\
「和を計算する」 \\
$1111_2 + 1101_2 = \fbox{1}1100_2 =  -4_{10}$ \\
(5ビット目の1は桁あふれで消滅)
\end{quote}
}}
\vspace{0.2cm}

以上のように,2の補数表現を使用すると負の数を含んだ足し算が簡単に
計算できます.ここでは省略しますが,引き算も同様に正負の区別をする
ことなく計算できます.

\vspace{0.8cm}
\fbox{\parbox{7cm}{
{\bf MSBとLSB}\\
 「2の補数表現を用いると,最上位ビットが1なら負の数と分かります.」
のように\underline{最上位ビット}と言う言葉をよく使います.\\
 「最上位ビット」と言うのは長いので,これを英語にした「Most Significant
Bit」の頭文字を取りMSBと略すことが良くあります.憶えておいて下さい.\\
 同様に「最下位ビット」のことは英語で「Least Significant Bit」なので,
略してLSBと呼びます.これも,憶えておきましょう.
}}

\vspace{0.3cm}
\begin{flushleft}
{\bf 問題}
\end{flushleft}
\begin{enumerate}
\item
次の数を「2の補数表現を用いた4ビット符号付き2進数」で書き表しなさい.\\
$3_{10}$,$-3_{10}$,$5_{10}$,$-5_{10}$,$6_{10}$,$-6_{10}$
\item
次の数を「2の補数表現を用いた8ビット符号付き2進数」で書き表しなさい.\\
$3_{10}$,$-3_{10}$,$8_{10}$,$-8_{10}$,\\
$30_{10}$,$-30_{10}$,$100_{10}$,$-100_{10}$
\item
次の「2の補数表現を用いた4ビット符号付き2進数」を10進数で書き表しなさい.\\
$1100_2$,$1011_2$,$0100_2$,$1101_2$
\item
次の2進数は「2の補数表現を用いた符号付き2進数」です.空欄を埋めなさい.

\begin{tabular}{ l c r  c c r }
1) &   & $0011~1100_2$ &    &   & $\fbox{   }_{10}$ \\
   & + & $0010~1101_2$ & → & + & $\fbox{   }_{10}$ \\
\cline{2-3} \cline{5-6}
   &   & $\fbox{    }_2$ & ~ & + & $\fbox{   }_{10}$ \\
   &   &                     &   &   &                      \\
2) &   & $0110~0100_2$ &    &   & $\fbox{   }_{10}$ \\
   & + & $1000~0001_2$ & → & + & $\fbox{   }_{10}$ \\
\cline{2-3} \cline{5-6}
   &   & $\fbox{    }_2$ & ~ & + & $\fbox{   }_{10}$ \\
   &   &                     &   &   &                      \\
3) &   & $1110~0100_2$ &    &   & $\fbox{   }_{10}$ \\
   & + & $0100~0001_2$ & → & + & $\fbox{   }_{10}$ \\
\cline{2-3} \cline{5-6}
   &   & $\fbox{    }_2$ & ~ & + & $\fbox{   }_{10}$ \\
   &   &                     &   &   &                      \\
4) &   & $1110~0100_2$ &    &   & $\fbox{   }_{10}$ \\
   & + & $1100~0001_2$ & → & + & $\fbox{   }_{10}$ \\
\cline{2-3} \cline{5-6}
   &   & $\fbox{    }_2$ & ~ & + & $\fbox{   }_{10}$ \\
%   &   &                     &   &   &                      \\
\end{tabular}
\end{enumerate}

\section{小数の表現}

この節では,小数を2進数で表現する方法を紹介します.

\subsection{2進数による小数の表現}

小数を含む数を,
コンピュータ内部で2進数を用いて表現する方法は何種類かあります.
ここでは,その中でも最も基本的な,固定小数点形式を紹介します.

これまでの2進数は,暗黙のうちに小数点が右端にあると考えました.
小数点の位置を右端以外だと約束すれば,小数を含む数の表現も可能になります.

\vspace{0.2cm}
\fbox{\parbox{6.5cm}{\small
例 4ビットで小数点が2桁目なら
\begin{quote}
$00.00_2 = 0.0_{10}$  \\
$00.01_2 = 0.25_{10}$  \\
$00.10_2 = 0.5_{10}$  \\
$00.11_2 = 0.75_{10}$  \\
$01.00_2 = 1.0_{10}$  \\
$01.01_2 = 1.25_{10}$  \\
$01.10_2 = 1.5_{10}$  \\
$01.11_2 = 1.75_{10}$  \\
$10.00_2 = 2.0_{10}$  \\
...\\
$11.11_2 = 3.75_{10}$
\end{quote}
}}
\vspace{0.2cm}

小数点を中心に,左は1桁毎に2倍,右は1桁毎に$1/2$倍になります.
(10進数では,左は1桁毎に10倍,右は1桁毎に$1/10$倍になりましたね.)

\subsection{10進数との相互変換}
\begin{enumerate}
\item
2進数から10進数への変換 \\
整数の場合と同様に桁の重みを加算すれば10進数に変換できます.

\fbox{\parbox{6.5cm}{\small
例 2進10進変換
\begin{quote}
$10.01_2 = 2 + 1/4 = 2.25_{10}$
\end{quote}
}}

\item
10進数から2進数への変換 \\
整数の場合は,$1/2$倍することで右にシフト(桁移動)し,小数点を横切った
$0/1$を判定しました.(小数点を$1$が横切ると,余りが出ていました.)

小数点以下の数値は,$2$倍することで左にシフトし,小数点を横切った
$0/1$を判定すれば2進数に変換できます.
\begin{center}
\fbox{\parbox{7cm}{
{\bf 例 小数を表す10進数から2進数への変換}\\

\begin{tabular}{ r l  l r}
\multicolumn{1}{c}{2進数} &  ~~~$\times 2$は~~~ & \multicolumn{2}{c}{10進数}\\
                $0.101_2$ &        左シフ       &           & $0.625$ \\
\multicolumn{1}{c}{$\swarrow$} &   トと同じ     & $\times$  &     $2$ \\
\cline{3-4}
                $1.010_2$      &                &           & $\underline{1}.250$ \\
&&&\\
\multicolumn{1}{c}{2進数} &  ~~~$\times 2$は~~~ & \multicolumn{2}{c}{10進数}\\
                $0.010_2$ &        左シフ       &           & $0.250$ \\
\multicolumn{1}{c}{$\swarrow$} &   トと同じ     & $\times$  &     $2$ \\
\cline{3-4}
                $0.100_2$      &                &           & $\underline{0}.500$ \\
&&&\\
\multicolumn{1}{c}{2進数} &  ~~~$\times 2$は~~~ & \multicolumn{2}{c}{10進数}\\
                $0.100_2$ &        左シフ       &           & $0.500$ \\
\multicolumn{1}{c}{$\swarrow$} &   トと同じ     & $\times$  &     $2$ \\
\cline{3-4}
                $1.000_2$      &                &           & $\underline{1}.000$ \\
\end{tabular}
\\ \\
10進数で計算したとき,小数点を横切って整数部に出てきた数を
小数点の右に順番に並べる.
\begin{center}
\Large{$0.\underline{101}_2$}
\end{center}
}}
\end{center}

\item 整数部と小数部の両方がある場合 \\
整数部分と小数部分を分離して,別々に計算します.
\end{enumerate}

%\vspace{0.3cm}
\begin{flushleft}
{\bf 問題}
\end{flushleft}
\begin{enumerate}
\item
次の2進数を10進数に変換しなさい.
\begin{enumerate}
\item $0.1001_2$
\item $0.0101_2$
\item $11.11_2$
\end{enumerate}
\item
次の10進数を2進数に変換しなさい.
\begin{enumerate}
\item $0.0625_{10}$
\item $0.1875_{10}$
\item $0.4375_{10}$
\end{enumerate}
\item
次の10進数を2進数に変換しなさい.(難しい問題)
\begin{enumerate}
\item $0.8_{10}$
\item $0.7_{10}$
\end{enumerate}
\end{enumerate}

\section{文字の表現}
\label{char}

この節では,文字を2進数で表現する方法を紹介します.
(文字情報を表しているビットの読み方を約束する.)

\subsection{文字コード}
文字の場合,数値のような規則性を期待することはできません.
そこで,使用する文字の一覧表を作成し,1文字毎に対応する2進数(コード)を
決めます.
この一覧表のことを「文字コード表」と呼びます.
文字コード表を各自(コンピュータメーカ等)が勝手に定義すると,
コンピュータの間でのデータ交換に不便です.
そこで,規格として制定してあります.

\subsection{ASCIIコード}
\label{ascii}
ASCII(American Standard Code for Information Interchange)コードは,
1963年にアメリカ規格協会(ANSI)が定めた情報交換用の文字コードです.
英字,数字,記号等が含まれます.
現在のパーソナルコンピュータ等で使用される文字コードは,ASCIIコードが
基本になっています.

\begin{center}
{\bf ASCII文字コード表}
\vspace{0.2cm}
\fbox{\parbox{7cm}{
\epsfxsize=6.5cm
%\epsfysize=22.5cm
\myepsfbox{ascii.pdf}
}}
\end{center}

文字コード表で 00H 〜 1FH と 7FH は,機能(改行等)を表す特殊な文字に
なっています.20H(SP)は空白を表す文字です.

ASCII文字コードは7ビットで表現されます.しかし,コンピュータの内部では
1バイト(8ビット)単位の方が扱いやすいので,最上位に$0_2$を補って8ビット
で扱うことがほとんどです.

\subsection{JIS文字コード}
日本では JIS(Japan Industrial Standard:日本工業規格)の一部として,
8ビットコード(英数記号+カナ)と,16ビットコード
(英数記号+カナ+ひらがな+カタカナ+漢字...)が定められています.

JIS 8ビットコードは,ASCIIコードに半角カタカナを追加したものです.
記号,数字,英字の部分は同じ並びになっています.

\section{補助単位}

長さや重さを書くとき,1,000m を 1km, 1,000g を 1kg のように書きました.
この k のような記号を補助単位と呼びます.

コンピュータの世界でよく使用される補助単位には,k(キロ=$10^3$),
M(メガ=$10^6$),G(ギガ=$10^9$),T(テラ=$10^{12}$)等があります.
(1,000倍毎に補助単位があります.)
パソコンのカタログに「CPUのクロックは 1GHz」のような記述を見かけますね.

記憶容量を表す場合の補助単位は,$1,000$の代わりに$2^{10} = 1,024$を使用します.
1,000 も 1,024 も k で表現すると紛らわしいので,
k の代わりに ki と書き表すこともあります.
つまり,$1kB =1kiB = 2^{10}B = 1,024B$,
$1MB =1MiB = 2^{20}B = 1,048,576B$となります.(「B」はバイトを表す.)

「H.D.D.の容量は500GB」のように表示されている場合は,
記憶容量を表しているので$G = 2^{30}$で表示されています.
同じことを「H.D.D.の容量は500GiB」のように表示することもあります.

\vspace{0.2cm}
\fbox{\parbox{7cm}{\small
{\bf 補助単位まとめ}\\
\begin{tabular}{r l l | r l l}
\multicolumn{3}{c|}{一般的に} &
\multicolumn{3}{c}{記憶容量} \\
\multicolumn{1}{c}{値} &
\multicolumn{1}{c}{記号} &
\multicolumn{1}{c|}{読み方} &
\multicolumn{1}{c}{値} &
\multicolumn{1}{c}{記号} &
\multicolumn{1}{c}{読み方} \\
$10^3$   & $k$ & キロ   & $2^{10}$ & $ki$ & キビ \\
$10^6$   & $M$ & メガ   & $2^{20}$ & $Mi$ & メビ \\
$10^9$   & $G$ & ギガ   & $2^{30}$ & $Gi$ & ギビ \\
$10^{12}$& $T$ & テラ   & $2^{40}$ & $Ti$ & テビ \\
\end{tabular}
}}

\section{コンピュータの基本回路}

前の節までで,コンピュータ内部での情報の表現方法を勉強しました.
コンピュータの内部では,どんな情報も電気の「ON/OFF」で表現しているのでしたね.

この節では,電気の「ON/OFF」を使用して,
計算や記憶をする回路を勉強します.

\subsection{論理演算と論理回路}

論理(Yes/No,真/偽,True/False)を対象とする演算(計算)のことを
\underline{論理演算}と呼びます.
論理の「真/偽」をビットの「1/0」に対応させることにより,
論理も電気の「ON/OFF」で表現することができます.

ここでは,論理演算と,論理演算をする回路を紹介します.
論理演算をする回路のことを\underline{論理回路}と呼びます.
ここで紹介する論理回路が,
コンピュータを構成する基本回路になります.

\subsection{基本的な論理回路}

基本的な論理回路を4種類,
組合せてできたものを2種類,紹介します.

\begin{enumerate}
\item 論理積(AND) \\
二つのビットを入力し,両方が1のときだけ出力が1になるような論理演算です.
\begin{center}
\epsfxsize=6.5cm
%\epsfysize=22.5cm
\myepsfbox{and.pdf}
\end{center}

\item 論理和(OR) \\
二つのビットを入力し,どちらかが1のとき出力が1になるような論理演算です.
\begin{center}
\epsfxsize=6.5cm
%\epsfysize=22.5cm
\myepsfbox{or.pdf}
\end{center}

\item 否定(NOT) \\
一つのビットを入力し,入力と逆の論理を出力する論理演算です.
\begin{center}
\epsfxsize=6.5cm
%\epsfysize=22.5cm
\myepsfbox{not.pdf}
\end{center}

\item 排他的論理和(XOR) \\
二つのビットを入力し,二つが異なるとき出力が1になるような論理演算です.
\begin{center}
\epsfxsize=6.5cm
%\epsfysize=22.5cm
\myepsfbox{xor.pdf}
\end{center}

\item NAND \\
否定(NOT)と論理積(AND)を組み合わせた演算です.
\begin{center}
\epsfxsize=6.5cm
%\epsfysize=22.5cm
\myepsfbox{nand.pdf}
\end{center}

\item NOR \\
否定(NOT)と論理和(OR)を組み合わせた演算です.
\begin{center}
\epsfxsize=6.5cm
%\epsfysize=22.5cm
\myepsfbox{nor.pdf}
\end{center}

\end{enumerate}

\subsection{演算回路}

「基本的な論理回路」を組み合わせることにより,
足し算/引き算等のより複雑な演算を行う回路を実現できます.

\begin{enumerate}
\item 半加算器 \\
例えば,1ビットの足し算回路は次の例のように実現できます.
この例では A,Bの2ビットを入力し,和(S)と上の桁への桁上がり(C)を
出力する回路を示しています.
このような回路のことを「半加算器」と呼びます.

\vspace{0.2cm}
\fbox{\parbox{6.6cm}{
{\bf 例 1ビット半加算器}
\begin{center}
\epsfxsize=6.5cm
%\epsfysize=22.5cm
\myepsfbox{ha.pdf}
\end{center}
}}

\item 全加算器 \\
半加算器は桁上がりを出力しますが,
自分自身は下の桁からの桁上がりを入力することができません.
桁上がりの入力を備えた1ビット足し算器を「全加算器」と呼びます.

次の例のように,
全加算器は半加算器を組み合わせて実現することができます.

\vspace{0.2cm}
\fbox{\parbox{6.6cm}{
{\bf 例 1ビット全加算器}
\begin{center}
\epsfxsize=5.5cm
%\epsfysize=22.5cm
\myepsfbox{fa.pdf}
\end{center}
}}

\item nビット加算器 \\
半加算器と全加算器を組み合わせることで,
任意ビットの加算器を実現することができます.
次に4ビット加算器の例を示します.

\vspace{0.2cm}
\fbox{\parbox{6.6cm}{
{\bf 例 4ビット加算器}
\begin{center}
\epsfxsize=4.7cm
%\epsfysize=22.5cm
\myepsfbox{adder.pdf}
\end{center}
}}

\pagebreak
\item nビット1の補数器 \\
1の補数を作る回路です.
1の補数はビット反転によってできるので,NOTでできます.
次に,4ビットの例を示します.
簡単ですね.

\vspace{0.2cm}
\fbox{\parbox{6.6cm}{
{\bf 例 4ビット1の補数器}
\begin{center}
\epsfxsize=4.7cm
%\epsfysize=22.5cm
\myepsfbox{onesc.pdf}
\end{center}
}}

\item nビット2の補数器 \\
2の補数を作る回路です.
2の補数は,1の補数に1足すとできるので,
1の補数器の出力に半加算器を組み合わせるとできます.

\vspace{0.2cm}
\fbox{\parbox{6.6cm}{
{\bf 例 4ビット2の補数器}
\begin{center}
\epsfxsize=6cm
%\epsfysize=22.5cm
\myepsfbox{towsc.pdf}
\end{center}
}}

\end{enumerate}

演算回路を数種類紹介しました.
ここに紹介しませんでしたが,引き算回路等も同様に製作可能です.

\subsection{記憶回路}
「基本的な論理回路」を組み合わせることにより,
記憶機能を実現することもできます.
ここでは,最も簡単なRS-FF(RSフリップフロップ)を紹介します.

RS-FFはS(Set)とR(Reset)の二つの入力と,状態$(Q,\bar Q)$の出力を
持つ回路です.
S=0,R=1を入力することにより,回路はリセットされ出力Qは0になります.
S=1,R=0を入力することにより,回路はセットされ出力Qは1になります.
\( \bar Q \)には,常にQの否定が出力されます.

S=0,R=0を入力すると,回路は直前の値を「記憶」します.
S=1,R=1は入力してはいけない組合せです.

RS-FFは,回路をリセット/セットした後で「記憶」状態にすることにより,
「値を記憶する回路(=記憶回路)」として働きます.

\vspace{0.2cm}
\fbox{\parbox{6.6cm}{
{\bf RS-FF}
\begin{center}
\epsfxsize=6.5cm
%\epsfysize=22.5cm
\myepsfbox{rsff.pdf}
\end{center}
}}

\vspace{0.3cm}

\fbox{\parbox{7.0cm}{
{\bf 論理回路素子}

{\small
 回路を製作するには,基本的な論理回路を内蔵した集積回路(論理IC)を用いる
ことができます.

 様々な論理ICが販売されていますが,ここでは,74シリーズのものを紹介します.
下の例に示したのは,NANDゲートを四つ内蔵した7400と呼ばれるICです.
同様なICで,AND,OR,XOR,NOT等の回路を内蔵したものがあります.

 このようなIC同士を配線して組み合わせることにより,
本文で紹介した演算回路や記憶回路を実現することができます.

\vspace{0.2cm}
{\bf 論理IC} \\
%\begin{center}
\epsfxsize=6.5cm
%\epsfysize=22.5cm
\myepsfbox{lic.pdf}
%\end{center}
}}}
