\renewcommand{\myincludegraphics}[2]{\includegraphics[#2]{appD/#1}}

\newpage
\onecolumn
\chapter{参考資料}

TeCをもっと活用してもらうために,
拡張ボード,TeCの回路図,FPGAのピン配置表等を掲載します.
次のページから以下の資料を掲載します.

\section{TeC7基板回路図}
\figref{appD:kairo}にTeC7本体のプリント基板回路図を示します.

\section{TeC7ピン配置表}
\figref{appD:pin}にTeC7に搭載されている
Xilinx Spartan-6 FPGA (XC6LX9-2TQG144C)の
ピンがどのように利用されているかを示します.

\newpage
\myfigureNA{btph}{angle=90,scale=0.80}{TeC7a.pdf}{TeC7a基板回路図}{appD:kairo}

\newpage
\myfigureNA{btph}{angle=90,scale=0.85}{TeC7aPIN.pdf}{TeC7aピン配置}{appD:pin}
