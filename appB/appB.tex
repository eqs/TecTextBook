\renewcommand{\myepsfbox}[1]{\epsfbox{appB/#1}}

\newpage
\onecolumn
\chapter{TeCクロス開発環境}
\label{cross}
\section{始めに}
TeCクロス開発環境は,
教育用コンピュータTeC(Tokuyama kousen Educational Computer)の
機械語プログラムを開発するためのパソコン上で動作するプログラム群である.
クロス開発環境は,
クロスアセンブラtasm(tasm7:TeC7用,tasm6:TeC6用),
ダウンロードプログラムtsend(tsend7:TeC7用,tsend6:TeC6用)よりなる.

tasmは,ニーモニックで記述されたソースプログラムを解釈し,
アセンブルリストと機械語をファイルに出力する.出力された機械語はtsend
プログラムにより,シリアル通信を用いてTeCにダウンロードする.
tasmとtsendを使用することにより,
TeCのプログラムをパソコン上でクロス開発することができる.
その様子を図に示す.

\begin{figure}[bthp]
\begin{center}
\epsfxsize=10cm
\myepsfbox{syori.pdf} \\
処理の流れ
\end{center}
\end{figure}

\section{アセンブルコマンド}
tasmは次の形式のコマンド行から起動され,
アセンブラソースプログラムを
アセンブルリスト(拡張子 .lst のファイルに出力)と
機械語(拡張子 .bin のファイルに出力)に変換(アセンブル)する.
なお,アセンブラソースプログラムは,TeC7では拡張子``.t7''のファイルに,
TeC6では拡張子``.t6''のファイルに格納する.
\begin{center}
{\tt tasm7 [-l nn] [-w nn] {\it source\_file} } \\
({\tt tasm6 [-l nn] [-w nn] {\it source\_file} })
\end{center}
{\it source\_file}は\ref{syn}章で説明する文法で記述されたアセンブラソース
ファイルの名前である.ソースファイル名の拡張子は``.t7''(``.t6'')
である必要がある.
オプションの意味は次の通りである.
\vspace{0.2cm}
\begin{center}
\begin{tabular}{ll} \hline
\multicolumn{1}{c}{\it オプション} & \multicolumn{1}{c}{\it 意味} \\ \hline
{\tt -l nn}  &  アセンブルリストの1ページをnn行にする. \\
{\tt -w nn}  &  アセンブルリストの1ページをnn桁にする. \\ \hline
\end{tabular}
\end{center}
\vspace{0.2cm}

\section{転送コマンド}
tsendはtasmが出力した機械語プログラムを
シリアル通信を用いてTeCに転送するコマンドである.
次の形式のコマンド行から起動する.

\begin{center}
{\tt tsend7  {\it bin\_file} {\it [com\_port]}} \\
({\tt tsend6  {\it bin\_file} {\it [com\_port]}})
\end{center}
{\it bin\_file}はtasmが出力した機械語プログラムファイルの名前である.
{\it [com\_port]}は,
TeCと接続されたシリアルポートを表すデバイスファイルの名前である.
tsendを起動すると画面に「TeCを受信状態にする」ように指示が表示される
ので,TeCのE0H番地に格納されているIPLプログラム
(\ref{iplsrc} IPLプログラム 参照)を起動する.
\vspace{0.2cm}

\section{アセンブラの文法}
\label{syn}
以下ではtasmの入力となるソースプログラムの文法について説明する.

\subsection{行フォーマット}
ソースファイルは,行の連なりからなるテキストファ
イルである.なお,tasmのソースファイルでは文字定数・文字列中を
除き英字大文字と小文字は区別されない.
空白はトークンの前後にいくつでも挿入することができる.
空白は,スペース(文字コード20H)とタブ(文字コード09H)の2種類である.
但し,\underline{行頭に空白を置いた場合は「ラベルが存在しない」ことを表す}.

\[ 行 = [ラベル] [命令とオペランド] [;コメント] \]

\subsection{ラベル}
行の最初の文字が英字の場合はその行にラベルがあるものとみなされる.
ラベルは英字で始まりその後に任意の長さの英数字が続く文字列である.
ラベルが無い行の場合は行を空白文字で始める.ラベルの長さに制限
はないが,アセンブルリストの見栄えから7文字以内を推奨する.
EQU命令のラベルを除き,その行のアドレスがラベルの値になる.

\subsection{疑似命令}
以下では,tasmが処理できる4種類の疑似命令について説明する.

\subsubsection{EQU命令}
EQU命令はラベルを定義するために使用する疑似命令である.
この命令にはラベルが必ず必要である.
命令行のラベルが式値で定義される.
式では,ラベルの前方参照はできない.
EQU命令の書式は次の通りである.

\[ EQU命令 = EQU 式 \]

\begin{center}
\tt\begin{tabular}{lll}
\hline
\multicolumn{3}{c}{\it EQU 命令の使用例} \\
\hline
\multicolumn{1}{c}{\it ラベル} & 
        \multicolumn{1}{c}{\it 命令} & \multicolumn{1}{c}{\it オペランド} \\
START & EQU & 03H \\
SIZE  & EQU & 100 \\
END   & EQU & START+SIZE \\
\hline
\end{tabular}
\end{center}
\vspace{0.2cm}

\subsubsection{ORG命令}
ORG命令は機械語を生成するアドレスを変更する疑似命令である.
機械語(データも含む)が生成される全ての行より前で使用された
場合は,単に次の命令のアドレスを変更する.
機械語が生成される行の途中で使用された場合は,
指定アドレスに達するまで,
ゼロで初期化された領域を生成する.
現在のアドレスより前のアドレスを指定することはできない.
ORG命令の式では,ラベルの前方参照はできない.
ORG命令の書式は次の通りである.
\[ ORG命令 = ORG 式 \]
\begin{center}
\tt\begin{tabular}{lll}
\hline
\multicolumn{3}{c}{\it ORG 命令の使用例} \\
\hline
\multicolumn{1}{c}{\it ラベル} & 
        \multicolumn{1}{c}{\it 命令} & \multicolumn{1}{c}{\it オペランド} \\
START & ORG & 03H \\
\hline
\end{tabular}
\end{center}
\vspace{0.2cm}

\subsubsection{DS命令}
DS命令は領域を定義するために使用する疑似命令である.
領域の値はゼロで初期化される.
DS命令の式では,ラベルの前方参照はできない.
DS命令の書式は次の通りである.
\[ DS命令 = DS 式 \]
\begin{center}
\tt\begin{tabular}{lll}
\hline
\multicolumn{3}{c}{\it DS 命令の使用例} \\
\hline
\multicolumn{1}{c}{\it ラベル} & 
        \multicolumn{1}{c}{\it 命令} & \multicolumn{1}{c}{\it オペランド} \\
WORK  & DS & 10 \\
\hline
\end{tabular}
\end{center}
\vspace{0.2cm}

\subsubsection{DC命令}
DC命令はメモリ上に任意の定数を出力するために使用する
疑似命令である.DC命令の書式は次の通りである.
\[ DC命令 =
 DC \left\{
  \begin{array}{c}
   式 \\
   文字列
  \end{array}  
 \right\}
[,\left\{
  \begin{array}{c}
   式 \\
   文字列
  \end{array}  
 \right\},~ ...~ ,
\left\{
  \begin{array}{c}
   式 \\
   文字列
  \end{array}  
 \right\}]
\]
\vspace{0.2cm}
\begin{center}
\tt\begin{tabular}{lll}
\hline
\multicolumn{3}{c}{\it DC 命令の使用例} \\
\hline
\multicolumn{1}{c}{\it ラベル} & 
        \multicolumn{1}{c}{\it 命令} & \multicolumn{1}{c}{\it オペランド} \\
DATA & DC & 1,2,3,4 \\
     & DC & 1+1 \\
CHA  & DC & 'a' \\
STR  & DC & "abc" \\
\hline
\end{tabular}
\end{center}
\vspace{0.2cm}

\subsection{機械語命令}
以下では,tasmが処理できる6グループの機械語命令について説明する.

\subsubsection{機械語命令1}
オペランドの無い1バイトの機械語命令である.
下表の命令がこのグループに属する.
\[ 機械語命令1 = 命令 \]
\begin{center}
\begin{tabular}{lll}
\hline
\multicolumn{2}{c}{\it 機械語命令1に属す命令} \\
\hline
\multicolumn{1}{c}{\it 命令} & \multicolumn{1}{c}{\it 意味} \\
NO & 何もしない \\
PUSHF & フラグをスタックに退避 \\
POPF & フラグをスタックから復旧 \\
EI & 割り込み許可 \\
DI & 割り込み禁止 \\
RET & サブルーチンから復帰 \\
RETI & 割り込みから復帰 \\
HALT & プログラム終了 \\
\hline
\end{tabular}
\end{center}
\vspace{0.2cm}

\begin{center}
\tt\begin{tabular}{lll}
\hline
\multicolumn{3}{c}{\it 機械語命令1の使用例} \\
\hline
\multicolumn{1}{c}{\it ラベル} & 
        \multicolumn{1}{c}{\it 命令} & \multicolumn{1}{c}{\it オペランド} \\
END  & HALT &  \\
\hline
\end{tabular}
\end{center}
\vspace{0.2cm}

\subsubsection{機械語命令2}
レジスタ指定付きの1バイトの機械語命令である.
レジスタとして.G0,G1,G2,SPが使用可能である.
下表の命令がこのグループに属する.
\[ 機械語命令2 = 命令 レジスタ \]
\begin{center}
\begin{tabular}{lll}
\hline
\multicolumn{2}{c}{\it 機械語命令2に属す命令} \\
\hline
\multicolumn{1}{c}{\it 命令} & \multicolumn{1}{c}{\it 意味} \\
SHLA & 左算術シフト \\
SHLL & 左論理シフト \\
SHRA & 右算術シフト \\
SHRL & 右論理シフト \\
PUSH & スタックにレジスタを退避 \\
POP  & スタックからレジスタを復旧 \\
\hline
\end{tabular}
\end{center}
\vspace{0.2cm}

\begin{center}
\tt\begin{tabular}{lll}
\hline
\multicolumn{3}{c}{\it 機械語命令2の使用例} \\
\hline
\multicolumn{1}{c}{\it ラベル} & 
        \multicolumn{1}{c}{\it 命令} & \multicolumn{1}{c}{\it オペランド} \\
L1  & SHRA &  SP \\
    & PUSH &  G0 \\
    & POP  &  G1 \\
\hline
\end{tabular}
\end{center}
\vspace{0.2cm}

\subsubsection{機械語命令3}
オペランドにレジスタと式を書く2バイトの機械語命令である.
レジスタとして.G0,G1,G2,SPが使用可能である.
下表の命令がこのグループに属する.
\[ 機械語命令3 = 命令 レジスタ,式 \]
\begin{center}
\begin{tabular}{lll}
\hline
\multicolumn{2}{c}{\it 機械語命令3に属す命令} \\
\hline
\multicolumn{1}{c}{\it 命令} & \multicolumn{1}{c}{\it 意味} \\
IN & 入出力ポートから読み込む \\
OUT & 入出力ポートに書き込む \\
\hline
\end{tabular}
\end{center}
\vspace{0.2cm}

\begin{center}
\tt\begin{tabular}{lll}
\hline
\multicolumn{3}{c}{\it 機械語命令3の使用例} \\
\hline
\multicolumn{1}{c}{\it ラベル} & 
        \multicolumn{1}{c}{\it 命令} & \multicolumn{1}{c}{\it オペランド} \\
SIO  & EQU &  03H \\
     & IN  &  G0,SIO \\
\hline
\end{tabular}
\end{center}
\vspace{0.2cm}

\subsubsection{機械語命令4}
レジスタとメモリをオペランドに指定できる命令で,メモリ指定に
イミディエイトモードが使用できるグループである.
レジスタとして.G0,G1,G2,SPが使用可能である.
インデクスレジスタとして.G1,G2が使用可能である.
下表の命令がこのグループに属する.
\[ 機械語命令4 = 命令 \left\{
  \begin{array}{c}
   レジスタ,式 [,インデクスレジスタ] \\
   レジスタ,#式 \\
  \end{array}  
 \right\}
 \]
\begin{center}
\begin{tabular}{lll}
\hline
\multicolumn{2}{c}{\it 機械語命令4に属す命令} \\
\hline
\multicolumn{1}{c}{\it 命令} & \multicolumn{1}{c}{\it 意味} \\
LD & レジスタにメモリオペランド値をロード \\
ADD & レジスタとメモリオペランド値の和 \\
SUB & レジスタとメモリオペランド値の差 \\
CMP & レジスタとメモリオペランド値を比較 \\
AND & レジスタとメモリオペランド値の論理積 \\
OR & レジスタとメモリオペランド値の論理和 \\
XOR & レジスタとメモリオペランド値の排他的論理和 \\
\hline
\end{tabular}
\end{center}
\vspace{0.2cm}

\begin{center}
\tt\begin{tabular}{lll}
\hline
\multicolumn{3}{c}{\it 機械語命令4の使用例} \\
\hline
\multicolumn{1}{c}{\it ラベル} & 
        \multicolumn{1}{c}{\it 命令} & \multicolumn{1}{c}{\it オペランド} \\
L1  & LD  & G0,DATA \\
    & LD  & G1,\#0 \\
    & SUB & G0,DATA,G1 \\
    & ADD & G1,\#1 \\
\hline
\end{tabular}
\end{center}
\vspace{0.2cm}

\subsubsection{機械語命令5}
機械語命令4と,ほぼ,同様であるが,イミディエイトモードが
使用できない機械語命令のグループである.
レジスタとして.G0,G1,G2,SPが使用可能である.
インデクスレジスタとして.G1,G2が使用可能である.
下表のようにST命令だけがこのグループに属する.
\[ 機械語命令5 = 命令 レジスタ,式 [,インデクスレジスタ]\]
\begin{center}
\begin{tabular}{lll}
\hline
\multicolumn{2}{c}{\it 機械語命令5に属す命令} \\
\hline
\multicolumn{1}{c}{\it 命令} & \multicolumn{1}{c}{\it 意味} \\
ST & レジスタの値をメモリオペランドに格納 \\
\hline
\end{tabular}
\end{center}
\vspace{0.2cm}

\begin{center}
\tt\begin{tabular}{lll}
\hline
\multicolumn{3}{c}{\it 機械語命令5の使用例} \\
\hline
\multicolumn{1}{c}{\it ラベル} & 
        \multicolumn{1}{c}{\it 命令} & \multicolumn{1}{c}{\it オペランド} \\
  & ST & G0,WORK \\
  & ST & G0,WORK,G1 \\
\hline
\end{tabular}
\end{center}
\vspace{0.2cm}

\subsubsection{機械語命令6}
メモリオペランドだけの機械語命令グループである.
下表のようにジャンプ・コール命令がこのグループに属する.
インデクスレジスタとして.G1,G2が使用可能である.
\[ 機械語命令6 = 命令 式 [,インデクスレジスタ]\]
\begin{center}
\begin{tabular}{lll}
\hline
\multicolumn{2}{c}{\it 機械語命令6に属す命令} \\
\hline
\multicolumn{1}{c}{\it 命令} & \multicolumn{1}{c}{\it 意味} \\
JMP & 必ずジャンプ \\
JZ  & Zフラグが'1'ならジャンプ \\
JC  & Cフラグに'1'ならジャンプ \\
JM  & Sフラグに'1'ならジャンプ \\
JNZ & Zフラグが'0'ならジャンプ \\
JNC & Cフラグに'0'ならジャンプ \\
JNM & Sフラグに'0'ならジャンプ \\
CALL & サブルーチン呼び出し \\
\hline
\end{tabular}

\vspace{0.2cm}
{\bf (JNZ,JNC,JNM命令はTeC6では使用できない.)}
\end{center}
\vspace{0.2cm}

\begin{center}
\tt\begin{tabular}{lll}
\hline
\multicolumn{3}{c}{\it 機械語命令6の使用例} \\
\hline
\multicolumn{1}{c}{\it ラベル} & 
        \multicolumn{1}{c}{\it 命令} & \multicolumn{1}{c}{\it オペランド} \\
  & JMP & LOOP \\
  & CALL & SUB1 \\
\hline
\end{tabular}
\end{center}
\vspace{0.2cm}

\newpage
\section{アセンブラの文法まとめ}
以下に文法を BNF 風にまとめる.\\
\vspace{0.3cm}

\begin{quote}
\small\tt\begin{tabular}{lll}
<プログラム>& ::= &{ <行> } EOF \\
<行>        & ::= &<命令行> | <空行> | <コメント行> \\
<命令行>    & ::= &<ラベル欄> [<命令欄>] [<コメント>] <改行> \\
<空行>      & ::= &<改行> \\
<コメント行>& ::= &<コメント> <改行> \\
<ラベル欄>  & ::= &<ラベル> | <空白> \\
<命令欄>    & ::= &<命令記述1> | <命令記述2> | <命令記述3> \\
              &     &  | <命令記述4> | <命令記述5> | <命令記述6> \\
              &     &  | <命令記述7> | <命令記述8> \\
<命令記述1>& ::= &<命令1> \\
<命令記述2>& ::= &<命令2> <レジスタ> \\
<命令記述3>& ::= &<命令3> <レジスタ> , <式> \\
<命令記述4>& ::= &<命令4> <レジスタ> , <アドレス1> \\
<命令記述5>& ::= &<命令5> <レジスタ> , <アドレス2> \\
<命令記述6>& ::= &<命令6> <アドレス2> \\
<命令記述7>& ::= &<命令7> <式> \\
<命令記述8>& ::= &<命令8> <値列> \\
<アドレス1>& ::= &# <式> | <式> [,<インデクス>] \\
<アドレス2>& ::= &<式> [,<インデクス>] \\
<値列>      & ::= &<拡張式> { ,<拡張式> } \\
<拡張式>    & ::= &<式> | <文字列> \\
<式>        & ::= &<項> |<項> + <式> | <項> − <式> \\
<項>     & ::= &<因子1> | <因子1> * <項> | <因子1> / <項> \\
<因子1>    & ::= &<因子> | +<因子> | −<因子> \\
<因子>      & ::= &<数値> | <ラベル> | ( <式> ) \\
<コメント>  & ::= &; { <文字1> } \\
<数値>      & ::= &<10進数> | <16進数> | <文字定数> \\
<10進数>  & ::= &<数字> { <数字> } \\
<16進数>  & ::= &<数字> { <16進数字> } H \\
<文字定数>  & ::= &'<文字2>' \\
<文字列>    & ::= &"{ <文字3> }" \\
<ラベル>      & ::= &<英字> { <英数字> } \\
<英数字>    & ::= &<英字> | <数字> \\
<文字1>    & ::= &<改行>以外の文字 \\
<文字2>    & ::= &<改行>,' 以外の文字 \\
<文字3>    & ::= &<改行>,” 以外の文字 \\
<16進数字>& ::= &0 | 1 | 2 | 3 | 4 | 5 | 6 | 7 | 8 | 9 \\
              &     & | A | B | C | D | E | F \\
<英字>      & ::= &A | B | C | D | E | F | G | H | I | J \\
              &     & | K | L | M | N | O | P | Q | R | S | T \\
              &     & | U | V | W | X | Y | Z | \_ \\
<数字>      & ::= &0 | 1 | 2 | 3 | 4 | 5 | 6 | 7 | 8 | 9 \\
<命令1>    & ::= &NO | PUSHF | POPF | EI | DI | RET |RETI | HALT \\
<命令2>    & ::= &SHLA | SHLL | SHRA | SHRL | PUSH | POP \\
<命令3>    & ::= &IN | OUT \\
<命令4>    & ::= &LD | ADD | SUB | CMP | AND | OR |XOR \\
<命令5>    & ::= &ST \\
<命令6>    & ::= &JMP | JZ | JC | JM | JNZ | JNC | JNM | CALL \\
<命令7>    & ::= &EQU | DS | ORG \\
<命令8>    & ::= &DC \\
<インデクス>& ::= &G1 | G2 \\
<レジスタ>  & ::= &G0 | G1 | G2 | SP \\
<改行>      & ::= &'$\backslash$n' \\
<空白>      & ::= &'~' |'$\backslash$t' \\
\end{tabular}
\end{quote}


\newpage
\section{ローマ字で名前をSIOに出力するプログラムの例}
\begin{quote}
\begin{verbatim}
ADR  CODE          Label   Instruction             Comment              Page(1)

02                 SIOD    EQU     02H            
03                 SIOS    EQU     03H            
00                 
00  17 00          START   LD      G1,#0          
02  19 15          L0      LD      G2,DATA,G1     
04  5B 00                  CMP     G2,#0          
06  A4 14                  JZ      END            
08  C0 03          L1      IN      G0,SIOS        
0A  63 80                  AND     G0,#80H        
0C  A4 08                  JZ      L1             
0E  CB 02                  OUT     G2,SIOD        
10  37 01                  ADD     G1,#1          
12  A0 02                  JMP     L0             
14  FF             END     HALT                   
15                 
15  73 69 67 65    DATA    DC      "sigemura"     
19  6D 75 72 61 
1D  0D 0A                  DC      0DH,0AH        
1F  00                     DC      0              ; 終わりの印
\end{verbatim}
\end{quote}

\section{アセンブル実行例}
TeC7を用い,
ローマ字で名前をSIOに出力するプログラムをアセンブル・実行した例を示す.
\begin{quote}
\begin{verbatim}
$ tasm7 sigemura.t7                            <--- アセンブル
アセンブル成功
結果は [sigemura.lst] と [sigemura.bin] に格納しました。

...省略...

$ tsend7 sigemura.bin                          <--- ダウンロード
TeC7を受信状態にして Enter キーを押して下さい。
[00][20][17][00][19][15][5b][00][a4][14][c0][03][63][80][a4][08][cb][02][37]
[01][a0][02][ff][73][69][67][65][6d][75][72][61][0d][0a][00]
$ screen /dev/tty.usbserial-xxxxxxxxx          <--- 通信ソフト起動
sigemura                                       <--- 実行結果
^A^\                                           <--- 通信ソフトの終了
[EOT]
\end{verbatim}
\end{quote}

\newpage
\section{IPLプログラム}\label{iplsrc}
TeCのROM に格納されているIPLプログラムのアセンブルリストを
参考のため以下に示す.
TeCのメモリ空間はE0H〜FFH番地がROMになっており,
以下のIPLプログラムが予め格納されている.
このIPLプログラムは,パソコン上でtasmを使用してクロス開発さ
れた機械語プログラムをシリアル入出力から受信してメモリに格納する.

\begin{quote}
\begin{verbatim}
ADR  CODE          Label   Instruction             Comment              Page(1)

02                 SIOD    EQU     02H            
03                 SIOS    EQU     03H            
E0                         ORG     0E0H           
E0  1F DC          IPL     LD      SP,#0DCH       ; 割込みベクタの直前がスタック
E2  B0 F6                  CALL    READ           ; ロードアドレスを入力
E4  D0                     PUSH    G0             
E5  D6                     POP     G1             
E6  B0 F6                  CALL    READ           ; 長さを入力
E8  D0                     PUSH    G0             
E9  DA                     POP     G2             
EA  A4 FF          LOOP    JZ      STOP           
EC  B0 F6                  CALL    READ           
EE  21 00                  ST      G0,0,G1        
F0  37 01                  ADD     G1,#1          
F2  4B 01                  SUB     G2,#1          
F4  A0 EA                  JMP     LOOP           
F6                 ;
F6  C0 03          READ    IN      G0,SIOS        
F8  63 40                  AND     G0,#40H        
FA  A4 F6                  JZ      READ           
FC  C0 02                  IN      G0,SIOD        
FE  EC                     RET                    
FF                 ;
FF  FF             STOP    HALT                   ; PC がゼロになって止まる
\end{verbatim}
{\bf(TeC6ではCALLの機械語がBCになる.)}
\end{quote}

\section{機械語プログラムファイル形式}
tasmが出力する機械語プログラムの形式は次の通りである.

\vspace{0.3cm}

%\begin{center}
{\small\bf
\begin{tabular}{|c|}
\hline
ロードアドレス(1バイト) \\
\hline
プログラム長(1バイト) \\
\hline
\\
...\\
機械語(1バイト以上) \\
...\\
\\
\hline
\end{tabular}
}
%\end{center}
