\chapter{はじめに}

\section{この科目で学ぶこと}
現代,コンピュータと呼ばれているものはスーパーコンピュータと呼ば
れる高価で高性能なものから,マイコンと呼ばれ炊飯器やエアコンに組
み込まれている小型のものまで,全て同じ原理に基づき動作しています.

この原理は,1946年に米国の数学者フォン・ノイマン(Von Neumann)
が提案したと言われています.現在のコンピュータの,ほぼ,全てのもの
は,ノイマンの提案した同じ原理を使用しており「ノイマン型コンピュー
タ」と呼ばれます.

「ノイマン型コンピュータ」が出現して既に約70年の時間が経過しま
したが,未だにこれを上回る「自動機械」の構成方法は発明されていま
せん.だから,パソコンのような一般の人がコンピュータだと考えてい
る装置だけでなく,炊飯器やエアコンの制御装置のようなコンピュータ
らしくない装置まで,「自動機械」が必要なところには全て「ノイマン型
コンピュータ」が使用されているのです.恐らく,あと数十年は,
「ノイマン型コンピュータ」の時代が続くでしょう.

この教科書では,教育用コンピュータ(TeC)を用いて,
この「ノイマン型コンピュータ」の動作原理を,
しっかり勉強できるようになっています.
ここでしっかり勉強しておけば,将来,どんな新型の
コンピュータに出会ったとしても,恐れることはありません.
所詮,「ノイマン型コンピュータ」の一種ですから,皆さんはその正体
を簡単に見抜くことができるはずです.

「ノイマン型コンピュータ」の原理をきちんと理解しておくことは,皆
さんにとって,寿命の長いエンジニアになるための大切なステップとなり
ます.しっかり,がんばって下さい.

\section{勉強の進め方}

この教科書は,高専や工業高校の1年生と2年生が,
教育用コンピュータTeCを教材に,
ノイマン型コンピュータの基本を学ぶことを想定して作ってあります.
1章から5章の途中までを1年で学び,
残りを2年生(半年〜1年間)で学びます.

1年生では,(1)コンピュータの内部で情報を表現する方法,
(2)TeCの組み立て,(3)TeCの基本的なプログラミングを学びます.
2年生では,TeCを用いた高度なプログラミングを学びます.

\section{教材用コンピュータ}

私達の身近にあるパーソナルコンピュータ(パソコン)や携帯電話のよう
なコンピュータシステムは,高度で複雑すぎて「ノイマン型コンピュー
タ」の原理を学ぶための教材としては適していません.

原理を学ぶのに適した単純で小さなコンピュータ(マイコン)を,
専用に開発しました.このマイコンはTeC(Tokuyama Educational
Computer : 徳山高専教育用コンピュータ)と呼ばれます.

TeCの特徴は単純で小さなことです.
単純で小さなことで,次のようなメリットがあります.

\begin{itemize}
\item[単純]
ノイマン型コンピュータの本質的な部分だけを勉強しやすい.
現在のパソコン等は,勉強するには難しすぎる.
\item[小型]
自宅に持ち帰り宿題ができる.
授業時間外でも,納得がいくまで色々と試してみることができる.
\end{itemize}

TeCには2003年から使用しているTeC6と,
2011年から使用を開始したTeC7の二種類があります.
本書ではTeC7を基本に解説しています.
TeC6とTeC7で異なる場合はTeC6の場合を追記するようにしています.

